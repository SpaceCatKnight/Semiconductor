\documentclass[a4paper,parskip,11pt, DIV12]{scrreprt}

\usepackage[ngerman]{babel} % FÌr Deutsch [english] zu [ngerman] Àndern. 
\usepackage[utf8]{inputenc}
\usepackage[T1]{fontenc}
\usepackage{blindtext}
\usepackage{graphicx}
\usepackage{subfigure}
%\renewcommand{\familydefault}{\sfdefault}
\usepackage{fancyhdr}
\usepackage{amsmath}
\usepackage{mdwlist} %Benötigt fÌr AbstÀnde in AufzÀhlungen zu löschen
\usepackage{here}
\usepackage{calc}
\usepackage{hhline}
\usepackage{marginnote}
\usepackage{chngcntr}
\usepackage{helvet}
\usepackage{tabularx}
\usepackage{titlesec} % TextÃŒberschriften anpassen

% \titleformat{Überschriftenklasse}[Absatzformatierung]{Textformatierung} {Nummerierung}{Abstand zwischen Nummerierung und Überschriftentext}{Code vor der Überschrift}[Code nach der Überschrift]

% \titlespacing{Überschriftenklasse}{Linker Einzug}{Platz oberhalb}{Platz unterhalb}[rechter Einzug]

\titleformat{\chapter}{\LARGE\bfseries}{\thechapter\quad}{0pt}{}
\titleformat{\section}{\Large\bfseries}{\thesection\quad}{0pt}{}
\titleformat{\subsection}{\large\bfseries}{\thesubsection\quad}{0pt}{}
\titleformat{\subsubsection}{\normalsize\bfseries}{\thesubsubsection\quad}{0pt}{}

\titlespacing{\chapter}{0pt}{-2em}{6pt}
\titlespacing{\section}{0pt}{6pt}{-0.2em}
\titlespacing{\subsection}{0pt}{5pt}{-0.4em}
\titlespacing{\subsubsection}{0pt}{-0.3em}{-1em}

%\usepackage[singlespacing]{setspace}
%\usepackage[onehalfspacing]{setspace}

\usepackage[
			%includemp,				%marginalien in Textkörper einbeziehen
			%includeall,
			%showframe,				%zeigt rahmen zum debuggen		
			marginparwidth=25mm, 	%breite der marginalien
			marginparsep=5mm,		%abstand marginalien - text
			reversemarginpar,		%marginalien links statt rechts
			%left=50mm,				%abstand von Seitenraendern
%			top=25mm,				%
%			bottom=50mm,
			]{geometry}		

%Bibliographie- Einstellungen
\usepackage[babel,german=quotes]{csquotes}
\usepackage[
   backend=bibtex8, 
   natbib=true,
    style=numeric,
    sorting=none
]{biblatex}
\bibliography{Quelle}
%Fertig Bibliographie- Einstellungen

\usepackage{hyperref}

\newcommand{\footnoteremember}[2]{
  \footnote{#2}
  \newcounter{#1}
  \setcounter{#1}{\value{footnote}}
}
\newcommand{\footnoterecall}[1]{%
  \footnotemark[\value{#1}]
}

\begin{document}
\selectlanguage{ngerman}
\begin{titlepage}
\begin{figure}[h]
\hfill
\subfigure{\includegraphics[scale=1]{uzh}}
\end{figure}
\vspace{1 cm}
\textbf{\begin{huge}Praktikumsbericht Festkörperphysik
\end{huge}}\\
\noindent\rule{\textwidth}{1.1 pt} \\

\begin{Large}\textbf{Widerstandsmessung an einem Halbleiter}
\end{Large}\\ 
\normalsize 
\par
\begingroup
\leftskip 0 cm
\rightskip\leftskip
\textbf{Modul:}\\ PHY210 \\ \\
\textbf{Assistent:}\\ Kay Waltar\\ \\
\textbf{Studenten:}\\Nora Salgo, Manuel Sommerhalder, Fabian Stäger\\ \\
\textbf{Datum des Versuchs:}\\ 16.06.2017 \\ \\
\par
\endgroup
\clearpage



\end{titlepage}


%Start Layout
\pagestyle{fancy}
\fancyhead{} 
\fancyhead[R]{\small \leftmark}
\fancyhead[C]{\textbf{Halbleiter} } 
\fancyhead[L]{\includegraphics[height=2\baselineskip]{uzh}}

\fancyfoot{}
\fancyfoot[R]{\small \thepage}
\fancyfoot[L]{}
\fancyfoot[C]{}
\renewcommand{\footrulewidth}{0.4pt} 

\addtolength{\headheight}{2\baselineskip}
\addtolength{\headheight}{0.6pt}


\renewcommand{\headrulewidth}{0.6pt}
\renewcommand{\footrulewidth}{0.4pt}
\fancypagestyle{plain}{				% plain redefinieren, damit wirklich alle seiten im gleichen stil sind (ausser titlepage)
\pagestyle{fancy}}

\renewcommand{\chaptermark}[1]{ \markboth{#1}{} } %Das aktuelle Kapitel soll nicht Gross geschriben und Nummeriertwerden

\counterwithout{figure}{chapter}
\counterwithout{table}{chapter}
%Ende Layout

\tableofcontents

\chapter{Physikalischer Hintergrund}
\label{ch:Physik}

Für das Experiment wird der Widerstand einer Siliziumprobe bei verschiedenen Temperaturen gemessen. Bei reinem Silizium handelt es sich um einen intrinsischen Halbleiter. Somit besteht die Ladungsträgerdichte zu gleichen Teilen aus einem Elektronenanteil $n$ und einem Lochanteil $p$.
Die elektrische Leitfähigkeit eines Halbleiters ist gegeben durch die Formel\footnoteremember{kittel}{Introduction to solid state physics / Charles Kittel.—8th ed.}
\begin{equation}
\label{eq:sigma}
\sigma = ne\mu_e + pe\mu_h ,
\end{equation}
wobei $\mu_e$ und $\mu_h$ jeweils die Mobilität der Elektronen- bzw. Loch-Beiträge ist. Diese ist leicht temperaturabhängig. Die Elektronenladungsdichte $n$ errechnet sich aus der Formel\footnoterecall{kittel}
\begin{equation}
n = \int_{E_c}^{\infty} D_e(E) f_e(E) dE 
\end{equation}
mit der Zustandsdichte\footnoterecall{kittel}
\begin{equation}
D_e(E) = \frac{1}{2 \pi^2} \left(\frac{2 m_e}{\hbar^2}\right)^{3/2} (E - E_c)^{1/2}
\end{equation}
($m_e$ ist die effektive Masse, $\hbar$ die reduzierte Planck-Konstante und $E_c$ die niedrigste Energie des Leitungsbandes) und der Fermi-Dirac-Statistik\footnoterecall{kittel}
\begin{equation}
f(E,T) = \frac{1}{\exp \left(\frac{\mu - E}{k_B T}\right) + 1}
\end{equation}
($\mu$ ist hier das chemische Potential), was sich mit der Näherung $\mu - E >> k_B T$ integrieren lässt zu\footnoterecall{kittel}
\begin{equation}
n = 2 \left(\frac{m_e k_B T}{2 \pi \hbar^2}\right)^{3/2} \exp \left(\frac{\mu - E_c}{k_B T}\right).
\end{equation}
Ganz analog errechnet sich die Lochdichte mit derselben Näherung zu\footnoterecall{kittel}
\begin{equation}
p = 2 \left(\frac{m_h k_B T}{2 \pi \hbar^2}\right)^{3/2} \exp \left(\frac{E_v - \mu}{k_B T}\right)
\end{equation}
($E_v$ ist die höchste Energie des Valenzbandes). Das Produkt von $n$ und $p$ ist somit nur von der Temperatur abhängig:
\begin{equation}
np = 4 \left(\frac{k_B T}{2 \pi \hbar^2}\right)^{3} (m_e m_h)^{3/2} \exp \left(\frac{-E_g}{k_B T}\right)
\end{equation}
Dabei ist $E_g = E_c - E_v$ die Energiebandlücke. Da die beiden Ladungsdichten im intrinsischen Fall gleich gross sind, lassen sie sich durch Folgende Formel in Abhängigkeit der Bandlücke und der Temperatur beschreiben:
\begin{equation}
n = p = 2 \left(\frac{k_B T}{2 \pi \hbar^2}\right)^{3/2} (m_e m_h)^{3/4} \exp \left(\frac{-E_g}{2 k_B T}\right)
\end{equation}
Der spezifische Widerstand ist somit nach einsetzen in Formel \ref{eq:sigma}
\begin{equation}
\rho = \frac{1}{\sigma} = \frac{1}{2 (\mu_e + \mu_h)e} \left(\frac{k_B T}{2 \pi \hbar^2}\right)^{-3/2} (m_e m_h)^{-3/4} \exp \left(\frac{E_g}{2 k_B T}\right)
\end{equation}
Da der elektrische Widerstand mit einem geometrischen Faktor direkt proportional zum spezifischen Widerstand ist, wird dieser somit
\begin{equation}
\label{eq:Endloesung}
R = A(T) T^{-3/2} \exp \left(\frac{E_g}{2 k_B T}\right)
\end{equation}
mit $A(T)$ als leicht temperaturabhängigem Proportionalitätsfaktor. 

\chapter{Experimenteller Aufbau}
Die Silizium-Probe ist auf einem isolierenden Probeträger aus Keramikmaterial angebracht. An diesem zylinderförmigen Träger sind die Kontakte für die Vierpunktmessung des Widerstandes angebracht, sowie ein Thermoelement, welches die Temperatur der Probe messen soll. Der Träger wird von der Heizung umschlossen. Im Innenraum der Heizung herrscht ein Vakuum um die Wärmekonvektion und Oxidation zu unterbinden. Dieser Unterdruck von $10^-5$ mBar wird mit einer Turbomolekularpumpe erzeugt. An der Heizung ist ein zweites Thermoelement für den Temperaturregler angebracht. Dieser Regler steuert die Heizleistung, seine Funktionsweise wird in Kapitel \ref{PID_Regler}  genauer beschrieben. Die Messdaten und die Daten des Reglers können mit einem LabVIEW Programm überprüft und gespeichert werden.

BILD
%\begin{figure}[htbp]
%\centering
%\includegraphics[width=0.9\textwidth]{}
%\caption{Aufbaud des Experiments}
%\label{Aufbau}
%\end{figure}


\section{PID-Regler}
\label{PID_Regler}
Der Temperaturregler ist ein PID-Regler (Proportional Integral Differential Regler). Dieser Regler basiert auf Rückkopplung. Es wird ein IST-Wert, in unserem Fall die aktuelle Temperatur der Heizspirale und ein SOLL-Wert, die gewünschte Temperatur, verglichen. Der PID-Regler steuert den Heizstrom als Funktion der Differenz zwischen IST- und SOLL-Wert. Er wird eingesetzt um potentielle Störungen zu glätten und eine stabile Temperaturregelung zu erhalten. Die drei Teile des PID-Reglers haben verschiedene Funktionen.

Proportional-Teil reagiert sehr schnell auf Temperaturänderung da er proportional zur Differenz des IST- und SOLL-Wertes ist. Mit einem Proportional-Teil alleine wird die gewünschte Temperatur aber nie erreicht.\\
Der Integral-Teil reagiert als Funktion der Differenz des Mittelwerts und des SOLL-Wertes. Dieser Teil reguliert also langsamer als der Proportional-Teil. Störungen werden ausgeglichen und der SOLL-Wert wird tatsächlich erreicht.\\
Der Differential-Teil extrapoliert den Temperaturverlauf und reagiert als Funktion der Differenz zwischen dem extrapolierten IST-Wert und dem SOLL-Wert. Dieser Teil reguliert schneller als der Proportional-Teil und ist somit sehr instabil gegenüber Störungen, eine Übersteuerung ist möglich.
In unserem Versuch bleibt der Differential-Teil darum ausgeschaltet. 


\section{Thermoelement}
Ein Thermoelement besteht aus einem Stromkreis mit zwei verschiedenen Metallen A und B. Wenn zwischen den Kontaktstellen eine Temperaturdifferenz existiert, entsteht gemäss dem Seebeck-Effekt eine elektrische Spannung 
$$ U = \int_{T_{1}}^{T_{2}} S_{B}(T) - S_{A}(T) \, dT $$
wobei die Seebeck-Koeffizienten $S_{A}$ und $S_{B}$ temperaturabhängige Materialeigenschaften mit der Einheit V$/$K sind. Mithilfe einer Spannungstabelle (siehe Anhang auf Seite \pageref{Kacktabelle}) kann der gemessenen Spannung eine Temperatur zugeordnet werden. In diesem Experiment wurde ein NiCr-Ni-Thermoelement verwendet.

\section{Vierpunkt-Widerstandsmessung}
Um den Widerstand der Probe möglichst genau zu messen, wird die Vierpunkt-Methode verwendet. Dabei erfolgt die Stromzufuhr und die Spannungsmessung über seperate Leitungen. Die Spannung wird hochohmig direkt ab der Probe abgegriffen. So kann der Widerstand im Gegensatz zur Zweipunkt-Methode ohne Verfälschungen der Messung durch die Widerstände der Kabel und Kontakte gemessen werden.

\newpage

\chapter{Messdaten}

\section{Temperaturverlauf}
Während des Versuchs wurde die Temperatur in der Heizspule und an der Probe gemessen. In Abbildung \ref{temp_time} ist ersichtlich, dass die Probentemperatur der Reglertemperatur im Aufwärmprozess hinterherhinkt. Für die Auswertung wird nur die Probentemperatur berücksichtigt.
%
\begin{figure}[htbp]
\centering
\includegraphics[width=0.9\textwidth]{temp_time.png}
\caption{Temperaturverlauf über die Zeit}
\label{temp_time}
\end{figure}
%

\section{Temperaturabhängigkeit des elektrischen Widerstands}
Die Siliziumprobe wurde zuerst auf $500^{\circ}$C erwärmt und dann wieder auf Raumtemperatur abgekühlt. Während des ganzen Prozesses wurde der elektrische Widerstand mithilfe der Vierpunkt-Messmethode ermittelt. In der Abbildung \ref{res_temp} ist ersichtlich, dass der Widerstandsverlauf bei Aufwärm- (blau) und Abkühlprozess (grün) leicht unterschiedlich ausfällt.
%
\begin{figure}[htbp]
\centering
\includegraphics[width=0.9\textwidth]{res_temp.png}
\caption{elektrischer Widerstand des Halbleiters}
\label{res_temp}
\end{figure}
%


\chapter{Auswertung}

Formel \ref{eq:Endloesung} in Kapitel \ref{ch:Physik} vereinfacht sich für T hinreichend klein zu:
\begin{equation}
\label{eq:Rexp}
R \approx B \exp \left(\frac{E_g}{2 k_B T}\right)
\end{equation}
mit einer thermostatischen Proportionalitätskonstante $B$, da die Temperaturabhängigkeit der Exponentialfunktion sehr gross gegenüber dem Beitrag von $T^{-3/2}$ und den Mobilitäten $\mu_e$ und $\mu_h$ ist.
\\
Der Logarithmus der Gleichung \ref{eq:Rexp} ist dann:
\begin{equation}
\label{eq:Rlog}
\ln(R) = \ln (B) + \frac{E_g}{2 k_B} T^{-1}
\end{equation}

Aufgrund der linearen Abhängigkeit von $T^{-1}$ ist es sinnvoll, die Messwerte in einem Plot mit $T^{-1}$ in der x-Achse und ln$(R)$ in der y-Achse darzustellen. Der dadurch entstandene Graph ist grösstenteils näherungsweise linear. Bei kleinen Temperaturen bzeziehungsweise grossem $T^{-1}$ scheitert die Näherung, die in Formel \ref{eq:Rexp} gemacht wurde. Die Steigung $a$ eines linearen Fits durch den näherungsweise linearen Bereich hängt aufgrund von Gleichung \ref{eq:Rlog} direkt mit der Bandlücke zusammen:
\begin{equation}
\label{eq:bandgap}
a = \frac{E_g}{2 k_B} \quad \Rightarrow \quad E_g = 2 k_B a
\end{equation}


Für beide Teile der Messung, Aufwärm- (Abbildung \ref{temp_heat}) und Abkühlprozess (Abbildung \ref{temp_cool}), wurde ein solcher linearer Fit durchgeführt. Dies führte zu den folgenden Resultaten:

%
\begin{center}
\begin{tabular}{lll}
					& Aufwärmen 					& Abkühlen\\
	\hline
	Steigung 			& $a = (5854\pm7)$K			& $a = (6061\pm1)$K\\
	y-Achsenabschnitt	& $b = -4.716\pm0.012$			& $b = -5.130\pm0.003$\\
	Bandlücke			& $E_{g} = (1.0089\pm0.0012)$eV	& $E_{g} = (1.0446\pm0.0002)$eV\\
	
%
\end{tabular}
\end{center}
%

Hier gilt es zu beachten, dass die angegebenen Unsicherheiten auf der Bandlücke nur dem Fehler auf dem Fit entsprechen. Die effektive Messunsicherheit muss deutlich grösser eingeschätzt werden (siehe Kapitel \ref{ch:Diskussion}).

\begin{figure}[h]
\centering
    	\includegraphics[width=0.8\textwidth]{temp_heat.png}
	\caption{Aufwärmprozess}
	\label{temp_heat}
\end{figure}

\begin{figure}[h]
\centering
    	\includegraphics[width=0.8\textwidth]{temp_cool.png}
	\caption{Abkühlprozess}
	\label{temp_cool}
\end{figure}

\clearpage

\chapter{Diskussion}

Da der Abkühlprozess sich über längere Zeit hinzog, ist anzunehmen, dass die Temperaturverteilung der Probe näher an einem thermodynamischen Gleichgewicht ist als im Falle des Aufwärmens. Die dadurch entstandene Abweichung in der Temperaturabhängigkeit des Widerstandes wird in Abbildung
\ref{res_temp} deutlich. Daraus lässt sich schliessen, dass die Messung während des Abkühlens die Präzisere von beiden ist. Die aus dieser Messung resultierende Bandlücke ist \begin{center}
$E_g$ = 1.04 eV.
\end{center} Dieser weicht um 6.3\% vom Literaturwert\footnote{Streetman, Ben G.; Sanjay Banerjee (2000). Solid State electronic Devices. 5th ed.} $E_g$ = 1.11 eV (bei 302 K) ab und ist auch erwartungsgemäss näher an jenem als der beim Aufwärmprozess errechnete Wert von $E_g$ = 1.01 eV. \\
\\
Fehlerquellen, die für die Abweichung vom Literaturwert verantwortlich sind, wären einerseits die Vernachlässigung der weiteren temperaturabhängigen Beiträge in Formel \ref{eq:Endloesung}, andererseits auch die Annahme, dass es sich bei der verwendeten Siliziumprobe um einen rein intrinsischen Halbleiter handelt. Zusätzlich zu möglichen Verunreinigungen bei der Herstellung des Präparats wurde diese durch unzählige Pratikumsversuche wahrscheinlich leicht beschädigt, schliesslich können Oxidationsprozesse auch beim angelegten Vakuum nicht vollständig ausgeschlossen werden. Auch beim langsamen Abkühlen ist die Temperaturverteilung der Siliziumprobe nie ganz konstant, was die Widerstandsmessung zusätzlich verfälscht. Aufgrund der variierenden Temperaturverteilung und der Tatsache, dass die Messung der Probentemperatur aussen am Probenträger durchgeführt wurde, ist auch bei der Temperaturmessung eine Messunsicherheit zu erwarten. Zu dieser kommt auch hinzu, dass beim Errechnen der Temperaturwerte aus der elektrischen Spannung am Thermoelement auf Seite \pageref{Kacktabelle} die Raumtemperatur mangels Thermometer abgeschätzt wurde. Letztlich ist die Grösse der Bandlücke selbst nicht gänzlich unabhängig von der Temperatur und ist folglich anhand der Beobachtung der Temperaturabhängigkeit des elektrischen Widerstandes nur mit begrenzter Genauigkeit messbar. \\
\\
Um diese Fehler zu minimieren, könnte das Experiment an einem neu hergestellten, monokristallinen Siliziumpräparat durchgeführt werden. Die Temperaturverteilung der Probe wäre zudem umso näher an einem thermodynamischen Gleichgewicht, wenn die Ofentemperatur während des Abkühlens sehr langsam und kontinuierlich nach unten geregelt werden würde. Ein Thermometer im Praktikumsraum würde ebenfalls zur Fehlerreduktion beitragen. Um die Bandlücke noch genauer zu messen, könnte man anstatt der Widerstandsmessung die optische Absorption bei verschiedenen Temperaturwerten messen. Diese Messmethode hätte den Vorteil, die Temperaturabhängigkeit der Bandlücke korrekt wiedergeben zu können.

\clearpage



\chapter{Appendix}
\section{Spannungstabelle Thermoelement}
\label{Kacktabelle}
\begin{figure}[htbp]
\centering
\includegraphics[width=1\textwidth]{spannungstabelle.png}
\label{temp_cool}
\end{figure}
%
Aus der Tabelle \ref{Kacktabelle} ist ersichtlich, dass der Zusammenhang zwischen Temperatur $T$ und der Thermospannung $V$ linear ist. Es gilt
\begin{equation}
T = c_1 V + c_2
\end{equation}
wobei beim NiCr-Cr Thermoelement
\begin{align*}
c_1 &= 24.3\\
c_2 &= 1.06
\end{align*}
 
 Diese Funktion gilt für die Vergleichsstellentemperatur $T_0 = 0^{\circ}$ C. Für abweichende Umgebungstemperatur $T_1$, in unserem Fall $26^{\circ}$ C, muss die Spannung entsprechend angepasst werden.
 
 
 \begin{equation}
 T = c_1 (V+ \frac{T_1-c_2}{c_1}+c_2= c_1 V + T_1
 \end{equation}
 
 

%\renewcommand{\bibname}{Quellenverzeichnis}
%\printbibliography

\end{document}
